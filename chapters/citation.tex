\newpage
\begin{center}
    \sanhao\heiti
    \textbf{参考文献}    
\end{center}
\addcontentsline{toc}{section}{参考文献}
列出作者直接阅读过或在正文中引用过的文献资料。撰写论文时,需注意引用权威和最新的文献。

参考文献需在引文右上角用方括号“[]”标明序号,如“基本机构[1]”,并在参考文献中列出。每一条参考文献著录均以“.”结束。参考文献要另起一页,一律放在正文之后,不得放在各章节之后。例如\citejournal{example1}

参考文献采用顺序编码制,需符合《信息与文献 参考文献著录规则》(GB/T 7714-2015)规范要求,文献类型和标识代码为:普通图书[M]、会议录[C]、汇编[G]、报纸[N]、期刊[J]、学位论文[D]、报告[R]、标准[S]、专利[P]、数据库[DB]、计算机程序[CP]、电子公告[EB]、档案[A]、舆图[CM]、数据集[DS]、其他[Z]。

参考文献中主要责任者的个人作者采用姓在前名在后的著录形式,当作者不超过3个时,全部照录。超过3个,著录的前3个作者其后加“,等”(,et al)。欧美著者的名可用缩写字母,缩写名后省略缩写点,姓和缩写名全大写。用汉语拼音书写的人名,姓全大写,名可缩写,取每个汉字拼音的首字母。

参考文献为五号宋体,英文及数字为五号Times New Roman字体,两端对齐。参考文献中的标点符号均为英文标点,常用的参考文献著录项目和格式示例如下:

% \bibliographystyle{gbt7714-numerical}
% \bibliography{citation}

\bibliographystylejournal{gbt7714-numerical}
\bibliographyjournal{citation}

% 本项目在cls文件中利用multibib为每一个提到的类别都设置了一个新的引用项目,感觉不一定都用的上